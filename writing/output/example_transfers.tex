We next show that an empirical analysis can still be AMIP-non-robust even after
outliers are removed. To that end, we apply our techniques to examine the
robustness of the main analysis from \citet{angelucci2009indirect}, one of the
flagship studies showing the impact of cash transfers on ineligible
(``non-poor'') households in the same villages, also known as ``spillover
effects.'' The authors trimmed the consumption outcome for the non-poor
households due to concerns about the influence of the largest values. Yet while
the analysis on the poor households is quite robust, the analysis on the
non-poor households---whom the trimming protocol actually affects---is much more
sensitive.

%%
\subsubsection{Background and replication}
%%

\citet{angelucci2009indirect} employ a randomised controlled trial to study the
impact of Progresa, a social program giving cash gifts to eligible poor
households in Mexico. The randomization occurs at the village level. So one can
estimate both a main effect on the poor households selected to receive Progresa
and also the impact on the non-eligible ``non-poor'' households located in the
same villages as Progresa-receiving poor households.

The main results of the paper show that there are strong positive impacts of
Progresa on total household consumption measured as an index both for eligible
poor households and for the non-eligible households; see Table 1 of
\citet{angelucci2009indirect}. The variable $\texttt{C\_ind}_{it}$ denotes total
household consumption for household $i$ in time period $t$. Values of
$\texttt{C\_ind}_{it}$ above 10,000 are removed; such households are, by
definition, non-poor. The authors study three different time periods separately
to detect any change in the impact between the short and long term. They
condition on a large set of variables (a household poverty index, land size,
head of household gender, age, whether the household speaks an indigenous
language, and literacy; at the locality level, a poverty index, and the number
of households) to help ensure a fair comparison between households in the
treatment and control villages. In this case these controls are important; the
effects on the ``non-poor'' households are significant at the 5\% level when the
controls are included, but they are only significant at the 10\% level in a
simple regression on a dummy for treatment status.

The full data for the paper is available on the website of the \emph{American
Economic Review} thanks to the open-data policies of the journal and the
authors. We can successfully replicate the results of this analysis with the
controls and without, and we proceed with the controls in our present analysis
in accordance with the original authors' preferred specification. We consider
the time periods indexed as $t=8,9,10$ in the dataset provided, though we note
that the authors do not rely on the results at $t = 8$ as the roll-out was still
ongoing. We employ $K$ control variables, where $X_{itk}$ is the $k$-th variable
for household $i$ in period $t$. Then we run the following regression:
%
\begin{align*}
%
\texttt{C\_ind}_{it} = \beta_0 + \beta_1 \texttt{treat}_{poor,i} + \beta_2
\texttt{treat}_{nonpoor,i} + \sum_{k = 1}^K \beta_{2+k}X_{itk} + \epsilon_{it}.
%
\end{align*}
%
Here, $\texttt{treat}_{poor,i}$ refers to an interaction between the treatment
indicator and an indicator for being a poor household; correspondingly,
$\texttt{treat}_{nonpoor,i}$ is an interaction between the treatment indicator
and an indicator for being a non-poor household.  We are able to exactly
replicate the results of Table 1 of \citet{angelucci2009indirect}, which
exhibits positive effects of cash transfers.

%%
\subsubsection{AMIP Sensitivity Results}

\CashTransfersResultsTable{}
%%
We apply our methodology to assess how many data points one need remove to
change the sign, the significance, or to generate a significant result of the
opposite sign to that found in the full sample. We focus on the latter two time
periods, as households had received only partial transfers in the first time
period, but we show all three in order to replicate Table 1 from the original
paper. \tableref{cash_transfers_re_run_table} shows our results. Focusing on
periods 9 and 10, we find that the inferences on the direct effects on the poor
households are quite robust, but the inferences on the indirect effects are less
so. For the analysis of the poor, one typically needs to remove much more than
1\% of the sample to change conclusions. For the analysis of the non-poor, we
can remove less than 0.1\% of the data to change conclusions. In fact, we can
remove only 3 data points to change the significance status for both $t = 9$ and
$t = 10$.

We again check the quality of our approximation. The ``Refit Estimate'' column
in \tableref{cash_transfers_re_run_table} shows the results of manually
re-running each analysis after removing the implicated data points. In most
cases the AMIP correctly identifies a combination of data points that can make
the claimed changes to the conclusions of the study. Although there are a few
cases where re-running the analysis fails to produce the predicted statistically
significant sign change, the observed changes are still large enough to be of
practical interest.  Furthermore, it is likely that the removal of a few
additional points would in fact produce the desired statistically significant
sign reversals.

Finally, we note that these results constitute an illustration of how gross
error robustness is distinct from AMIP robustness (see
\secpointref{influence_function_for_real}{gross_errors}). Recall that
\citet{angelucci2009indirect} removed (non-poor) datapoints for which
consumption was greater than 10,000. By removing outliers of the consumption
variable in this way, the authors of this study made what is typically
considered a conservative choice in view of classical robustness concerns about
gross error sensitivity. Yet, as we have shown in
\tableref{cash_transfers_re_run_table}, qualitative conclusions concerning the
non-poor households remain non-robust to the removal of a small number of
datapoints, which demonstrates empirically that one cannot necessarily make an
analysis AMIP-robust by simply trimming outliers.  Indeed, as we showed above in
\secref{influence_function_ols}, even perfectly specified OLS regressions with
no aberrant data points can be AMIP-non-robust if the signal to noise ratio is
too low.
