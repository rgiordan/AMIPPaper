
%\subsubsection{Bayesian Hierarchical Tailored Mixture Model}

In this section, we investigate a Bayesian hierarchical model, both
demonstrating that even Bayesian analyses can exhibit considerable AMIP
sensitivity, and showing an example of a parameter of interest for which our
linear approximation performs badly. We specifically focus on a variational
Bayes approximation to the tailored mixture model from
\citet{meager2020aggregating}.  One might hope that any of the following aspects
of the more complicated model might alleviate AMIP sensitivity: the use of
hierarchical Bayesian evidence aggregation, the regularization from
incorporation of priors, or the somewhat more realistic data-generating process
captured in this specific tailored likelihood. Indeed, the approach of
\citet{meager2020aggregating} was specifically motivated by the desire to
capture important features of the data-generating process such as heavier tails.
On the contrary, we find that the average  estimated effects of microcredit
remain sensitive according to the AMIP, as we did in the simpler models of
\secref{example_microcredit_linear}.  We also find that the linear approximation
that underlies the AMIP performs poorly when attempting to decrease a particular
hypervariance parameter, providing a concrete example of the limitations of our
methodology, particularly for parameters near the boundary of the set of their
allowable values.

%%
\subsubsection{Background}
%%

Following \citet{meager2020aggregating}, we fit a hierarchical model (hereafter
referred to as the ``microcredit model'') to all the data from the seven
microcredit RCTs. We model each outcome using a spike at zero and two lognormal
tail distributions, one for the positive realizations of profit and one for the
negative realizations. Within the model, microcredit can affect the proportion
of data assigned to each of these three components as well as affecting the
location and scale of the lognormal tails. There is a hierarchical shrinkage
element to the model for each parameter. The hypervariances of the treatment
effects are of particular interest because these capture heterogeneity in
effects across studies and offer information about the transportability of
results across settings.

% For the remainder of this section we will focus only on
% the treatment effect of microcredit on the location parameters of the
% positive and negative tail distributions, which are denoted $\tau_{+}$ and
% $\tau_{-}$ respectively.
The models in the original paper were fit via Hamiltonian Monte Carlo (HMC) with
the software package Stan \citep{carpenter2017stan}. It is possible to compute
the Approximate Maximum Influence Perturbation for HMC, or for any Markov Chain
Monte Carlo method, using the tools of Bayesian local robustness
\citep{gustafson2000local, giordano2018covariances}, but the sensitivity of
simulation-based estimators is beyond the scope of this paper. However, there
are ways to estimate Bayesian posteriors via Z-estimators, such as with
Variational Bayes (VB) techniques \citep{blei2016variational}.\footnote{The
Laplace approximation can also be expressed as a Z-estimator.} Specifically, we
fit the microcredit model using a variant of Automatic Differentiation
Variational Inference (ADVI) described in \citet[Section
5.2]{giordano2018covariances} (see also the original ADVI paper,
\citet{kucukelbir2017advi}).  Since the posterior uncertainty estimates of
vanilla ADVI are notoriously inaccurate, we estimated posterior uncertainty
using linear response covariances, again following \citet[Section
5.2]{giordano2018covariances}.\footnote{When forming the Approximate Most
Influential Set, we approximated the sensitivity only of the posterior means to
data removal; the linear response covariances were considered fixed.  However,
when we report the results of re-fitting the model, we did re-calculate the
linear response covariances at the new variational optimum.}  We verified that
the posterior means and covariance estimates produced by our variational
procedure and the corresponding estimates from running HMC with Stan were within
reasonable agreement relative to the posterior standard deviation.

%%
\subsubsection{AMIP Sensitivity Results}
%
We first consider the effect of microcredit on the location parameter of the
positive and negative tails of profit, given respectively by the parameters
$\tau_{+}$ and $\tau_{-}$.  Roughly speaking, $\tau_{+}$ and $\tau_{-}$ are both
estimating the effect of microcredit averaged across all of the seven countries
analyzed in \secref{example_microcredit_linear}.  Our point estimates for
$\tau_{+}$ and $\tau_{-}$ are given by their respective VB posterior means.  We
used the linear response covariance estimates to form a 95\% posterior credible
interval in place of confidence intervals, and consider a change ``significant''
if the posterior credible interval does not contain zero.

\MicrocreditMixtureResultsTable{}
%
\MicrocreditMixtureSdResultsTable{}

\tableref{mcmix_re_run_table} shows the sensitivity of inference concerning
$\tau_{+}$ and $\tau_{-}$.  We see that the microcredit model's estimates of the
average effectiveness of microcredit remain highly sensitive to the removal of
small percentages of the sample, despite being derived from a model that
accounts for non-Gaussian data shape and is regularised by the priors. This
sensitivity shows that Bayesian aggregation procedures do not necessarily
produce AMIP-robust estimates.

We next examine the sensitivity of the hypervariances, which measure the
variability of the effect of microcredit on these tails from country to country.  Specifically,
the parameters $\sigma_{\tau_{+}}^2$ and $\sigma_{\tau_{-}}^2$ represent the
between-country variances of the effect of microcredit on positive and negative
profit outcomes, respectively. The $\sigma$ parameter can be thought of as the
scale parameter analogue of the corresponding location parameter $\tau$ from
\tableref{mcmix_re_run_table}. The hypervariances are of particular practical
interest because they quantify how variable the effect of microcredit might
be; small values of the hypervariance imply that all countries respond
similarly to microcredit, whereas large values imply that one should not
necessarily extrapolate the efficacy of microcredit from one country to another.

In order to avoid the possibility of extrapolating to negative variances, we form
a linear approximation to our variational Bayes estimates of the posterior mean
of $\log \sigma$.  Since $\log \sigma$ is a scale parameter measuring the
variability from country to country of the effect of microcredit, its sign is
not particularly meaningful, nor is it particularly interesting to ask whether
its posterior credible interval contains zero.  Rather, we are interested in the
magnitude of $\log \sigma$. So, to investigate robustness, we use the AMIP to
check the approximate maximum change achievable in either direction (increasing
or decreasing the magnitude of $\log \sigma$) by removing 0.5\% of the sample,
about the same fraction of the data as could generate a ``significant'' sign
change for the $\tau_{\pm}$ parameters.

The results for the hypervariances, given in \tableref{mcmix_sd_re_run_table},
represent a useful demonstration of the limitations of our linear approximation.
We are able to find sets of datapoints which, when dropped, produce {\em
increases} in the hypervariances, though the our linear approximation is not
nearly as accurate as in the rest of our results above. When we attempted to
drop points in order to decrease the hypervariances, however, the linear
approximation failed utterly; the Approximate Most Influential Set designed to
produce a decrease in the hypervariances instead produced a large {\em
increase} upon refitting.

Given that the hypervariances are constrained to be positive, our failure to
produce large decreases may not be surprising. Note that the hypervariances'
posterior expectations began very small, and that decreasing them pushes the
posterior of the hypervariances closer to the boundary of the admissible space.
Though the log variance may in principle take arbitrarily negative values, it
nevertheless appears that the model exhibits strongly non-linear dependence on
the data weights for very small variances. Designing useful diagnostics for
detecting and explaining such deviations from nonlinearity in complex models is
an interesting avenue for future work. In the meantime,
\tableref{mcmix_sd_re_run_table} shows the importance, when possible, of
checking the accuracy of the AMIP predictions by refitting the model, and of
exercising caution when using the AMIP approximation near the boundary of the
parameter space.
