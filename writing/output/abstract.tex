% Study samples often differ from the target populations of inference and policy
% decisions in non-random ways.  Researchers typically believe that such
% departures from random sampling --- due to changes in the population over time and
% space, or difficulties in sampling truly randomly --- are small, and their
% corresponding impact on the inference should be small as well.  We might
% therefore be concerned if the conclusions of our studies are excessively
% sensitive to a very small proportion of our sample data.  We propose a method to
% assess the sensitivity of applied econometric conclusions to the removal of a
% small fraction of the sample.  Manually checking the influence of all possible
% small subsets is computationally infeasible, so we use an approximation to find
% the most influential subset.  Our metric, the ``Approximate Maximum Influence
% Perturbation,'' is based on the classical influence function, and is
% automatically computable for common methods including (but not limited to) OLS,
% IV, MLE, GMM, and variational Bayes.   We provide finite-sample error bounds on
% approximation performance.   At minimal extra cost, we provide an exact
% finite-sample lower bound on sensitivity.   We find that sensitivity is driven
% by a signal-to-noise ratio in the inference problem, is not reflected in
% standard errors, does not disappear asymptotically, and is not due to
% misspecification. While some empirical applications are robust, results of
% several influential economics papers can be overturned by removing less than 1\%
% of the sample.

% AER abstracts must be < 100 words
Study samples often differ from the target populations in non-random ways, so it
is important to assess whether small changes in the population would lead to
large changes in substantive conclusions. We propose a method to approximately
assess the sensitivity of empirical results to the removal of a small fraction
of the sample.  We show that such sensitivity is not reflected in standard
errors, does not disappear asymptotically, and is not due to misspecification.
Our method is automatically computable for many common estimators, and we show
that several influential economics studies are highly sensitive to small subsets
of their data.