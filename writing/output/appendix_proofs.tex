

\begin{lem}\lemlabel{normalized_sum_bound}
%
\sloppy
Let $\chi_1, \ldots, \chi_N$ be real-valued scalars with $\meann \chi_n^2 = 1$.
Then $\max_{\w \in W_\alpha} \meann \abs{\w_n - 1}\chi_n \le \sqrt{\alpha}$.
%
\begin{proof}
%
Without loss of generality, let the $\chi_n$ be unique (if they are not,
add an arbitrarily small amount of jitter to break ties), and let $q_{1 -
\alpha}$ denote their $\lceil (1 - \alpha) N \rceil$-th largest value.
The maximum $\max_{\w \in W_\alpha} \meann \abs{\w_n - 1} \chi_n$ is
achieved at $\w$ which sets to zero the weights of all $\{n : \chi_n \ge
q_{1-\alpha}\}$, so
%
\begin{align*}
%
\max_{\w \in W_\alpha} \meann \abs{\w_n - 1}\chi_n =
\meann \ind{\chi_n \ge q_{1 - \alpha}} \chi_n.
%
\end{align*}
%
Let $\hat{F}_{\chi}$ denote the empirical distribution on $\chi_n$
conditional on the data $\d_n$, and note that $q_{1 - \alpha}$ is fixed
in $\hat{F}_{\chi}$.  Applying Cauchy-Schwartz to the preceding display
with the distribution $\hat{F}_{\chi}$ gives
%
\begin{align*}
%
%\MoveEqLeft
\meann \ind{\chi_n \ge q_{1 - \alpha}} \chi_n
%\\&
\le \sqrt{\meann \ind{\chi_n \ge q_{1 - \alpha}}^2}
\sqrt{\meann \chi_n^2} =
\sqrt{\frac{\lfloor N \alpha \rfloor}{N}}
\le \sqrt{\alpha},
%
\end{align*}
%
since $\meann \chi_n^2 = 1$ and at most $\lfloor \alpha N \rfloor$
points are greater than $q_{1-\alpha}$.
%
\end{proof}
%
\end{lem}


%%%%%%%%%%%%%%%%%%%%%%%%%%%%%%%%%%%%%%%%%%%%%%%%%%%%%%%%%%%%%%%%%%%%%%%%%
%%%%%%%%%%%%%%%%%%%%%%%%%%%%%%%%%%%%%%%%%%%%%%%%%%%%%%%%%%%%%%%%%%%%%%%%%
%%%%%%%%%%%%%%%%%%%%%%%%%%%%%%%%%%%%%%%%%%%%%%%%%%%%%%%%%%%%%%%%%%%%%%%%%

The following lemma shows that \assuref{ij_assu} satisfies Condition 1 of
\citet{giordano:2019:swiss}.

\begin{lem}\lemlabel{alpha_complexity}
%
Let $W_\alpha^* := \{ \onevec + t (\w - \onevec): \w \in W_\alpha, t \in [0, 1] \}$
Under \assuref{ij_assu},
%
\begin{align*}
%
\max_{\w \in W_\alpha^*}
\sup_{\theta \in \thetadom}
    \vnorm{\meann (\w_n - 1) G(\theta, \d_n)}_1   \le&
                \sqrt{D} \cgh \sqrt{\alpha} \quad\textrm{and}\\
%
\max_{\w \in W_\alpha^*}
\sup_{\theta \in \thetadom}
    \vnorm{\meann (\w_n - 1) H(\theta, \d_n)}_1   \le&
                \sqrt{D} \cgh \sqrt{\alpha}.
%
\end{align*}
%
\begin{proof}
%
We prove the result for $G(\theta, \d_n)$; the proof for $H(\theta, \d_n)$
follows analogously.  By the triangle inequality and the relationship between
$\vnorm{\cdot}_2$ and $\vnorm{\cdot}_1$,
%
\begin{align*}
%
\vnorm{\meann (\w_n - 1) G(\theta, \d_n)}_1 \le
\sqrt{D} \cgh \meann \abs{\w_n - 1} \frac{\vnorm{G(\theta, \d_n)}_2}{\cgh}.
%
\end{align*}
%
Apply \lemref{normalized_sum_bound} with $\chi_n := \frac{\vnorm{G(\theta,
\d_n)}_2}{\cgh}$ to control the maximum of the sum over $W_\alpha$.  Finally,
the results extends to $W_\alpha^*$ since
%
\begin{align*}
%
\max_{\w \in W_\alpha^*} \meann \abs{\w_n - 1}\chi_n
=&
\max_{t \in [0,1]} \max_{\w \in W_\alpha}
    \meann \abs{t (\w_n - 1)} \chi_n =
\max_{\w \in W_\alpha} \meann \abs{(\w_n - 1)} \chi_n.
%
\end{align*}
%
\end{proof}
%
\end{lem}



%%%%%%%%%%%%%%%%%%%%%%%%%%%%%%%%%%%%%%%%%%%%%%%%%%%%%%%%%%%%%%%%%
%%%%%%%%%%%%%%%%%%%%%%%%%%%%%%%%%%%%%%%%%%%%%%%%%%%%%%%%%%%%%%%%%
%%%%%%%%%%%%%%%%%%%%%%%%%%%%%%%%%%%%%%%%%%%%%%%%%%%%%%%%%%%%%%%%%

We need the following lemma to extend the result of \citet{giordano:2019:swiss},
Theorem 1 to smooth functions.

\begin{lem}\lemlabel{derivative_smooth}
%
Let \assuref{ij_assu, thetafun_smooth} hold. For sufficiently small $\alpha$,
there exists a constant $C_b < \infty$ such that, for any $a \in \mathbb{R}^N$,
%
\begin{align*}
%
\max_{\w \in W_\alpha^*}
\vnorm{\left(
    \fracat{d \thetahat(\w)}{ d\w^T}{\w} -
    \fracat{d \thetahat(\w)}{ d\w^T}{\onevec}
    \right) a }_2
\le C_b \frac{\vnorm{a}_2}{\sqrt{N}} \sqrt{\alpha}.
%
\end{align*}
%
\begin{proof}
%
As in the proof of \thmref{thetafun_accuracy}, for the remainder of the proof
assume that $\alpha \le \frac{\Delta^2}{D\cgh^2}$, and observe that Assumptions
1-5 and Condition 1 of \citet{giordano:2019:swiss} are satisfied.
%
For the duration of this proof, define the shorthand notation
%
\begin{align*}
%
H(\w) := \meann \w_n H(\thetahat(\w), d_n)
\quad\textrm{and}\quad
G(\w) := \meann a_n G(\thetahat(\w), d_n).
%
\end{align*}
%
Then, by the indicated results from \citet{giordano:2019:swiss},
%
\begin{align*}
%
\MoveEqLeft
\vnorm{\left(
    \fracat{d \thetahat(\w)}{ d\w^T}{\w} -
    \fracat{d \thetahat(\w)}{ d\w^T}{\onevec}
    \right) a }_2
\\={}&
\vnorm{-H(\w)^{-1} G(\w) + H(\onevec)^{-1} G(\onevec)}_2
    \quad \textrm{(Proposition 4)}
\\\le{}&
\vnorm{-(H(\w)^{-1} - H(\onevec)^{-1} ) G(\w)}_2 +
\vnorm{ H(\onevec)^{-1} ( G(\onevec) - G(\w))}_2
%
\\\le{}&
\vnorm{-(H(\w)^{-1} - H(\onevec)^{-1} ) G(\w)}_2 + \cop \delta.
    \quad \textrm{(Condition 1, Assumption 2)}
%
\end{align*}
%
Then,
%
\begin{align*}
%
\MoveEqLeft
\vnorm{(H(\w)^{-1} - H(\onevec)^{-1} ) G(\w)}_2
\\={}&
\vnorm{ H(\w)^{-1} \left(H(\onevec) -  H(\w)\right) H(\onevec)^{-1} G(\w)}_2
\\\le{}&
2\cop^2 \vnorm{\left(H(\onevec) -  H(\w)\right) G(\w)}_2
\quad\textrm{(Assumption 2, Lemma 6)}
\\\le{}&
2\cop^2 \sqrt{D} (1 + D C_w L_h \cop) \delta \vnorm{G(\w)}_2
\quad\textrm{(Lemma 5, Matrix norms)}
\\={}&
2\cop^2 \sqrt{D} (1 + D C_w L_h \cop) \delta
     \vnorm{\meann a_n G(\thetahat(\w), d_n)}_2
\\\le{}&
2\cop^2 \sqrt{D} (1 + D C_w L_h \cop) \delta
     \cgh \frac{\vnorm{a}_2}{\sqrt{N}}.
\quad\textrm{(Assumption 3, Cauchy-Schwartz)}
%
\end{align*}
%
Combining, and using our \lemref{alpha_complexity} to give $\delta = \sqrt{D}
\cgh \sqrt{\alpha}$, gives the desired result.
%
\end{proof}
%
\end{lem}



%%%%%%%%%%%%%%%%%%%%%%%%%%%%%%%%%%%%%%%%%%%%%%%%%%%%%%%%%%%%%%%%%%%%%%%%%
%%%%%%%%%%%%%%%%%%%%%%%%%%%%%%%%%%%%%%%%%%%%%%%%%%%%%%%%%%%%%%%%%%%%%%%%%
%%%%%%%%%%%%%%%%%%%%%%%%%%%%%%%%%%%%%%%%%%%%%%%%%%%%%%%%%%%%%%%%%%%%%%%%%

\textbf{Proof of \thmref{thetafun_accuracy}.}
%
\sloppy For the duration of the proof, define the linear approximation
$\thetalin(\w) := \thetahat + \fracat{\dee\thetahat(\w)}{\dee\w^T}{\onevec}(\w -
\onevec)$. \Assuref{ij_assu} is equivalent to Assumptions 1-4 of
\citet{giordano:2019:swiss}, and \lemref{alpha_complexity} satisfies Condition 1
of \citet{giordano:2019:swiss} with $\delta = \sqrt{D} \cgh \sqrt{\alpha}$.
Assumption 5 of \citet{giordano:2019:swiss} is satisfied for $W_\alpha$ with
$C_w = 1$. Define, as in \citet{giordano:2019:swiss}, $\cij := 1 + D \lh \cop$
and $\Delta := \min\left\{\Delta_\theta \cop^{-1}, \frac{1}{2}\cop^{-1}\cij^{-1}
\right\}$, So Lemma~3 and Theorem~1 of \citet{giordano:2019:swiss} give,
respectively, that
%
\begin{align}
%
\max_{\w \in W_\alpha^*}
    \vnorm{\thetahat(\w) - \thetahat}_2 \le{}& \cop \sqrt{D} \cgh \sqrt{\alpha}
\quad \textrm{and}  \eqlabel{theta_diff_bound} \\
%
\alpha \le
\frac{\Delta^2}{D\cgh^2} \quad\Rightarrow\quad
%
\max_{\w \in W_\alpha^*}\vnorm{\thetalin(\w) - \thetahat(\w)}_2
    \le{}& 2 \cop^2 \cij D \cgh^2 \alpha. \eqlabel{theta_accuracy_bound}
%
\end{align}
%
For the remainder of the proof assume that $\alpha \le
\frac{\Delta^2}{D\cgh^2}$ so that \eqref{theta_accuracy_bound} applies.

For any $\w \in W_\alpha$, define $\omega(t) := \onevec + t (\w - \onevec)  \in
W_\alpha^*$. By the fundamental theorem of calculus,
%
\begin{align}\eqlabel{thetafun_integral}
%
\thetafun(\thetahat(\w), \w) - \thetafunhat = \int_0^1
    \fracat{\dee \thetafun(\omega(t))}{\dee t}{t}dt
=
\int_0^1
    \left( \fracat{\dee \thetafun(\omega(t))}{\dee t}{t} -
           \fracat{\dee \thetafun(\omega(t))}{\dee t}{1}\right)\dee t +
    \fracat{\dee \thetafun(\omega(t))}{\dee t}{1}.
%
\end{align}
%
where, by the chain rule,
%
\begin{align*}
%
\fracat{\dee \thetafun(\omega(t)))}{\dee t}{t}
={}&
\fracat{\partial \thetafun(\theta, \omega(t))}
       {\partial \theta^T}{\thetahat(\omega(t))}
\fracat{\dee \thetahat(\w)}{\dee \w^T}{\omega(t)} (\w - \onevec) +
\fracat{\partial \thetafun(\thetahat(\omega(t)), \w)}
       {\partial \w^T}{\omega(t)} (\w - \onevec).
%
\end{align*}

It will be useful to adopt a specific ``big O'' notation for the remainder of the
proof, by which we mean the following.  If we write $x = O(\sqrt{\alpha})$ for
some quantity $x$, we mean that there exists a constant $C$, available as a
closed-form function of constants defined in \assuref{ij_assu, thetafun_smooth},
such that $x \le C \sqrt{\alpha}$ for all $\alpha \le \frac{\Delta^2}{D\cgh^2}$.
An analogous notation meaning is given to $x = O(\alpha)$.  This ``big O''
notation can be manipulated in the usual ways \citep{bruijn:1981:asymptotic}.

To begin with, by definition of $W_\alpha$, we have $\max_{\w \in W_\alpha}
\meann (\w_n - 1)^2 = \frac{\lfloor \alpha N \rfloor}{N} \le \alpha$, so
$\max_{\w \in W_\alpha} \vnorm{(\w - \onevec) / \sqrt{N}}_2 \le \sqrt{\alpha}$.

Next, observe that \eqref{theta_diff_bound, theta_accuracy_bound} together imply
that
%
\begin{align*}
%
\max_{\w \in W_\alpha^*}\vnorm{\fracat{\dee\thetahat(\w)}{\dee\w^T}{\onevec}(\w -
\onevec)}_2
\le \max_{\w \in W_\alpha^*} \vnorm{\thetalin(\w) - \thetahat(\w)}_2 +
\max_{\w \in W_\alpha^*} \vnorm{\thetahat(\w) - \thetahat}_2
= O(\sqrt{\alpha}).
%
\end{align*}
%
By \lemref{derivative_smooth} below, we have that
%
\begin{align*}
%
\max_{t \in [0,1]}
\max_{\w \in W_\alpha^*}
\vnorm{\left(
    \fracat{\dee \thetahat(\w)}{ \dee \w^T}{\omega(t)} -
    \fracat{\dee \thetahat(\w)}{ \dee \w^T}{\onevec}
    \right) (\w - \onevec) }_2
\le C_b \frac{\vnorm{\w - \onevec}_2}{\sqrt{N}} \sqrt{\alpha}
= O(\alpha).
%
\end{align*}
%
Combining the previous two displays gives, by the triangle inequality, that
$\max_{t \in [0,1]} \max_{\w \in W_\alpha^*} \vnorm{\fracat{\dee \thetahat(\w)}{
\dee\w^T}{\omega(t)}(\w - \onevec)}_2 = O(\sqrt{\alpha})$.

Finally, by the Lipschitz property of the partial derivatives in
\assuref{thetafun_smooth}, we have that
%
\begin{align*}
%
\max_{t \in [0,1]} \vnorm{
    \fracat{\partial \thetafun(\theta, \omega(t))}
           {\partial \theta}{\thetahat(\omega(t))} -
   \fracat{\partial \thetafun(\theta, \onevec)}
          {\partial \theta}{\thetahat}
}_2
% \le L_\phi \left(
%     \vnorm{\thetahat(\omega(t)) - \thetahat}_2 +
%     \frac{\vnorm{\omega(t) - \onevec}_2}{\sqrt{N}}
={}& O(\sqrt{\alpha}) \quad \textrm{and}\\
%
\max_{t \in [0,1]} \sqrt{N} \vnorm{
    \fracat{\partial \thetafun(\thetahat(\omega(t)), \w)}
           {\partial \w}{\omega(t)} -
    \fracat{\partial \thetafun(\thetahat, \w)}
           {\partial \w}{\onevec}
}_2 ={}& O(\sqrt{\alpha}).
%
\end{align*}
%
Again, the triangle inequality with the boundedness of the partial
derivatives of $\phi$ at $\w = \onevec$ implies
%
\begin{align*}
%
\max_{t \in [0,1]} \vnorm{
    \fracat{\partial \thetafun(\theta, \omega(t))}
           {\partial \theta}{\thetahat(\omega(t))}
}_2 \quad\textrm{and}\quad
\max_{t \in [0,1]} \sqrt{N} \vnorm{
    \fracat{\partial \thetafun(\thetahat(\omega(t)), \w)}
           {\partial \w}{\omega(t)}
}_2
={}& O(\sqrt{\alpha}).
%
\end{align*}
%
Combining the above results gives that
%
\begin{align*}
%
\max_{t \in [0,1]} \vnorm{\fracat{d \thetafun(\omega(t))}{d t}{t}}_2
= O(\sqrt{\alpha})
\quad\textrm{and}\quad
\max_{t \in [0,1]} \vnorm{
    \fracat{\dee \thetafun(\omega(t))}{\dee t}{t} -
    \fracat{\dee \thetafun(\omega(t))}{\dee t}{1}
}_2  = O(\alpha),
%
\end{align*}
%
from which the desired conclusion follows by \eqref{thetafun_integral}.
%
\qed


%%%%%%%%%%%%%%%%%%%%%%%%%%%%%%%%%%%%%%%%%%%%%%%%%%%%%%%%%%%%%%%%%%%%%%%%%
%%%%%%%%%%%%%%%%%%%%%%%%%%%%%%%%%%%%%%%%%%%%%%%%%%%%%%%%%%%%%%%%%%%%%%%%%
%%%%%%%%%%%%%%%%%%%%%%%%%%%%%%%%%%%%%%%%%%%%%%%%%%%%%%%%%%%%%%%%%%%%%%%%%
