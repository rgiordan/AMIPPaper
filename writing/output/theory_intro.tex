We now establish the determinants and accuracy of AMIP robustness. We begin by
deriving the key quantities of AMIP robustness in the simple case of correctly
specified univariate OLS regression (\secref{influence_function_ols}). For this
simple case, we show with theory and simulations that AMIP robustness is not
necessarily driven by misspecification, that AMIP non-robustness does not vanish
asymptotically, and that AMIP robustness is distinct from standard errors. Next,
we formally extend these conclusions to general Z-estimators in
\secref{influence_function}.  Finally, in \secref{accuracy}, we establish
conditions under which the approximation is provably uniformly accurate for
small $\alpha$, both in finite sample and asymptotically.

We will see that a central equation in our understanding of AMIP robustness is
its decomposition into three key quantities: the signal, noise, and shape.
First, the \emph{signal} $\Delta$ is the size of change in our quantity of
interest that would reverse our substantive conclusion (see
\secref{function_examples} above). Large values of the signal $\Delta$ indicate
that large changes are needed to make a different decision.  Second, the
\emph{noise} $\inflscale$ is defined by
%
\begin{align}\eqlabel{inflscale_def}
%
\inflscale^2 := \meann (N \infl_n)^2
%
\end{align}
%
We call $\inflscale$ the noise because $\inflscale^2$ is typically a consistent
estimator of the variance of the limiting distribution of $\sqrt{N}
\thetafun(\thetahat)$, a fact that will follow below from the relationship
between AMIP robustness, robust standard error estimators, and the influence
function (see \secpointref{amip_decomposition}{noise} or, more
generally, \secpointref{influence_function_for_real}{scale_via_influence}).
Third, the \emph{shape} $\shape$ is defined as
%
\begin{align}\eqlabel{shape_def}
%
\shape := -\frac{1}{N}
\sum_{n=1}^{\lfloor \alpha N \rfloor} \frac{N \infl_{(n)}}{\inflscale}
\ind{\infl_{(n)} < 0},
%
\end{align}
%
where $\infl_{(n)}$ refers to the $n$-th order statistic of the influence
scores, and $\ind{\cdot}$ denotes the indicator function taking value $1$ when
its argument is true and $0$ otherwise. The shape $\shape$ depends in a
complicated way on the shape of the tail of the distribution of the influence
scores, but we show that $0 \le \shape \le \sqrt{\alpha(1 - \alpha)}$ with
probability one, and that $\shape$ converges in probability to a nonzero
constant under standard assumptions (see
\secpointref{amip_decomposition}{shape}).  Given these three quantities, we will
show in \secpointref{amip_decomposition}{amip_decomposition} that
%
\begin{align} \eqlabel{robustness_three_parts}
%
\textrm{An analysis is AMIP non-robust }\quad\Leftrightarrow\quad
\frac{\Delta}{\inflscale} \le \shape.
%
\end{align}
%
We refer to the quantity $\Delta / \inflscale$ as the \emph{signal-to-noise ratio}.
For a given $\alpha$, \eqref{robustness_three_parts} suggests that it is the
signal-to-noise ratio that primarily determines AMIP robustness. Additionally, this
decomposition allows us to succinctly compare AMIP robustness to standard errors
and gross-error robustness, as well as to analyze the large-$N$ behavior of
AMIP robustness.

This section will use the following notation.  Let the symbol $\plim$ denote
convergence in probability, and $\dlim$ denote convergence in
distribution, both as $N \rightarrow \infty$.  Let $\vnorm{\cdot}_{op}$ denote
the operator norm of a matrix.
