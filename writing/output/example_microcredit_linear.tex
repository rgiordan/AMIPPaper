We now show that even a simple 2-parameter linear model that performs a
comparison of means between the treatment and control group of a randomised
trial can be highly sensitive. To that end, we consider the analysis of seven
randomised controlled trials of expanding access to microcredit, first
aggregated in \citet{meager2019understanding}.  In
\secref{example_microcredit_hierarchical} below, we will consider a more
complicated Bayesian hierarchical model on the same data.

%%
\subsubsection{Background}
%%
Each of the seven microcredit studies was conducted in a different country, and
each study selected certain communities to randomly receive greater access to
microcredit. Researchers either built a branch, or combined building a branch
with some active outreach, or randomly selected borrowers among those who
applied. The selected studies are:
%
\citet{angelucci2015microcredit}, \citet{attanasio2015impacts},
\citet{augsburg2015impacts}, \citet{banerjee2015miracle},
\citet{crepon2015estimating}, \citet{karlan2011microcredit}, and
\citet{tarozzi2015impacts}.
%
Six of these studies were published in a special issue of the \emph{American
Economics Journal: Applied Economics} on microcredit. All seven studies together
are commonly considered to represent the most solid evidence base for
understanding the impact of microcredit.

We follow the original studies and \citet{meager2019understanding} in analyzing
the impact of access to microcredit as the treatment of interest. The studies
range in their sample sizes from around 1,000 households in Mongolia
\citep{attanasio2015impacts} to around 16,500 households in Mexico
\citep{angelucci2015microcredit}. We first focus on the headline results on
household business profit regressed on an intercept and a binary variable
indicating whether a household was allocated to the treatment group or to the
control group. For household $i$ in site $k$, let $Y_{ik}$ denote the profit
measured, and let $T_{ik}$ denote the treatment status. We estimate the
following model via OLS:
%
\begin{equation}\eqlabel{mc_regression}
	Y_{ik} = \beta_0 + \beta T_{ik} + \epsilon_{ik}.
\end{equation}

This regression model compares the means in the treatment and control groups and
estimates the difference as $\hat{\beta}$. We follow
\citet{meager2019understanding} in omitting the control variables or fixed
effects from the regressions in order to examine the robustness of this
fundamental procedure. But in principle this omission should make no difference
to the estimate $\hat{\beta}$, and indeed it does not
\citep{meager2019understanding}.\footnote{The omission may in principle make a
difference to the inference on $\beta$ by affecting the standard errors.
However, it turns out that in these studies the additional covariates make very
little difference to the standard errors. We also do not cluster the standard
errors at the community level for the same reason; the results are not
substantially changed. Running the regression above in each of the seven studies
delivers almost identical results to the preferred specification, as it should
if intra-cluster correlations are weak and covariates are not strongly predictive
of household profit.}

%%
\subsubsection{AMIP sensitivity results}
%
\MicrocreditProfitResultsTable{}
\MicrocreditTemptationResultsTable{}
%
%%
The sensitivity results for the linear regression of profit on microcredit
access appear in \tableref{mc_profit_results}. In all cases, by removing less
than 1\% of the data points can change either the sign or the significance. In
three of the studies, one can drop less than 1\% of the data points to generate
a result of the opposite sign that would be deemed significant at the 5\% level.
Mexico, the largest study, is the most sensitive: a single data point among the
16,561 households in Mexico determines the sign (as also discussed above in
\secref{linear_regression}). To produce a statistically significant result of
the opposite sign---that is, to turn Mexico's noisy negative result into a
``strong'' positive result---one need remove only 15 data points, less than
0.1\% of the sample. Mongolia, the smallest study in terms of sample size, is
among the most robust in terms of sign changes; it takes 2\% of the sample to
change the sign. Producing a significant result of the opposite sign also
requires more than 1\% removal in the Philippines, Bosnia, Ethiopia, and
Mongolia---whereas Mexico, India, and Morocco are more sensitive. We check the
performance of our approximation by manually re-running the analysis with the
data removed; the ``Refit Estimate'' column shows that the claimed reversal is
always achieved in practice for these analyses.

By comparing the results of the present section with those of
\secref{example_medicaid,example_transfers}, we can confirm the conclusion of
\secpointref{amip_robustness_breakdown}{amip_is_not_se} that standard errors
are, in general, distinct from AMIP sensitivity.  Despite the fact that original
estimates of \tableref{mc_profit_results} are statistically insignificant, some
of these non-significant results are more AMIP-robust than some of the
significant results in the Cash Transfers and Oregon Medicaid examples; consider
the ``Significant sign change'' result in the Philippines study, for example.

We next demonstrate that the AMIP sensitivity observed in
\tableref{mc_profit_results} cannot simply be ascribed to statistical
insignificance. To do so, we consider a different outcome with smaller variability
and show that it reveals a similar sensitivity to the profit outcome.
The variable we now consider is household consumption spending on temptation goods such
as alcohol, chocolate, and cigarettes, since the effect of microcredit on
temptation spending was estimated by \citet{meager2019understanding} with the
greatest precision of all six considered outcome variables.
\Tableref{mc_temptation_results} shows the results of applying the AMIP to  the
same regression given in \eqref{mc_regression}, but with temptation spending as the outcome. While
somewhat more robust than the profit analyses, the difference in the approximate
removal proportions in \tableref{mc_temptation_results} is not large.

Finally, one might be tempted to ascribe the AMIP-non-robust results in
\tableref{mc_profit_results} to outliers resulting from the heavy tails of the
household profit variable (a phenomenon well-documented by
\citet{meager2020aggregating}). However, as we discuss in
\secpointref{influence_function_for_real}{gross_errors} above, gross error
robustness is qualitatively distinct from AMIP robustness (see also the
discussion of outlier trimming at the end of \secref{example_transfers}).
Indeed, the more complex hierarchical model of the next section,
\secref{example_microcredit_hierarchical}, was designed precisely to accommodate
the heavy tail of the household profit variable, and yet---as we will
show---still exhibits a high degree of AMIP-sensitivity.
